\documentclass[11pt]{article}

\usepackage{setspace}
\usepackage[letterpaper,margin=1in]{geometry}
\usepackage[parfill]{parskip}
\newcommand{\tab}{\hspace*{2em}}

\title{Computational Medical Diagnosis}
\author{John Foley}
\date{10 April 2014}

\begin{document}
\maketitle

\thispagestyle{empty}

\begin{abstract}
  Computers are increasingly being used in the medical industry. They are used to provide structure and efficiency to record 
  management and patient organization to both hospitals and medical labs, but the intelligence and efficiency of computers
  is also advanced enough to be able to analyze massive quantities of information and produce intelligent, reasonable
  conclusions. Of course in the context of diagnosis and medication, the data can consist of human medical records and
  genetic sequencing of genetic material and conclusions be diagnoses and care planning. Analysis of this information is
  currently conducted by one or more highly trained doctors or medical experts. The cost of these trained professionals
  is relatively high compared to the time it takes to perform the analysis per patient. Using computers, with advanced
  algorithm and high level cognitive science research, we are able to replicate the team of specialists and come to the same
  results in significantly less time and hard work, freeing those professionals to be able to perform harder problems. Another
  advantage to have expert diagnostic systems is having available oversight for work done by human doctors. Research has shown
  a connection to linguistic expression and diagnostic correctness. A computer that is programmed to diagnose can evaluate
  a doctor's work and catch costly misdiagnoses before they can occur.  
\end{abstract}

\pagebreak

\begin{doublespace}
\section{Introduction}

\section{Computer Aided Medical Tools}

\subsection{Improved Imaging Analysis}

\section{Genetic Problem Solving}

\subsection{Quicker, Efficient Analysis}

\subsection{Solving Problems}

\section{Doctor Oversight}

\subsection{Linguistic Expression}

\section{Conclusion}


\pagebreak


\end{doublespace}

\nocite{*}

\bibliographystyle{plain}
\bibliography{master}

\end{document}