\documentclass[11pt]{article}

\usepackage[letterpaper,margin=1in]{geometry}

\title{Computational Medical Diagnosis}
\author{John Foley}
\date{13 April 2014}
\begin{document}
\maketitle

\thispagestyle{empty}

\begin{abstract}
  Agile Methodologies have taken the software engineering industry by storm, especially in this modern
  era of internet technology and application production. The Agile Manfiesto, published in 2001,
 introduced many terms and ideas to describe the current set of methodologies that we call Agile. There are many
 problems in software engineering that Agile Methodologies are designed to address, such as ability to react to 
 change and adapt, maintain communication with clients instead of renegotiating contract, and evolutionary development cycles
 instead of planned in phases. These solve problems that are inherent in software development, and are not handled well in past
 methodologies such as Waterfall. 
\end{abstract}

\section{Client Contact with Small, Cross-Functional Teams}

Communication with the client during the life of software is critically important. It ensures that the product matches
what the client is expecting. In the past, client requirements are constructed into a contract for the developer to create,
but that allows for divergence if every detail is ambiguous.  Agile Methodologies that teams and a client representative
create a relationship to stimulate communication.

\subsection{Client in the Room}
Agile Methodlogies advocate for someone with domain knowledge be present during development so that specific details and 
questions can be answered immediately. Of course this would be hard and costly to maintain, so good communication 
should be maintained in its stead.

\subsection{Small Teams and Scrums}
Teams are kept small in order to maintain inter-team communication. Internal workings are easier to maintain and manage,
as well as becoming more comfortable with whom a developer works with.

\subsection{Scrums and Meetings}
Scrums are quick, efficient meetings for
teams to keep on track. Scrums are held often and typically report driven.



\nocite{*}

\bibliographystyle{plain}
\bibliography{master}

\end{document}