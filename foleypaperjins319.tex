\documentclass[11pt]{article}

\usepackage{setspace}
\usepackage[letterpaper,margin=1in]{geometry}
\usepackage[parfill]{parskip}
\newcommand{\tab}{\hspace*{2em}}

\title{Computational Medical Diagnosis}
\author{John Foley}
\date{10 April 2014}

\begin{document}
\maketitle

\thispagestyle{empty}

\begin{abstract}
  Computers are increasingly being used in the medical industry. They are used to provide structure and efficiency to record 
  management and patient organization to both hospitals and medical labs, but the intelligence and efficiency of computers
  is also advanced enough to be able to analyze massive quantities of information and produce intelligent, reasonable
  conclusions. Of course in the context of diagnosis and medication, the data can consist of human medical records and
  genetic sequencing of genetic material and conclusions be diagnoses and care planning. Analysis of this information is
  currently conducted by one or more highly trained doctors or medical experts. The cost of these trained professionals
  is relatively high compared to the time it takes to perform the analysis per patient. Using computers, with advanced
  algorithm and high level cognitive science research, we are able to replicate the team of specialists and come to the same
  results in significantly less time and hard work, freeing those professionals to be able to perform harder problems. Another
  advantage to have expert diagnostic systems is having available oversight for work done by human doctors. Research has shown
  a connection to linguistic expression and diagnostic correctness. A computer that is programmed to diagnose can evaluate
  a doctor's work and catch costly misdiagnoses before they can occur.  
\end{abstract}

\pagebreak

\begin{doublespace}
\section{Introduction}

    \tab Computational intelligence and capability is quickly becoming noticeable in today's society as a helpful tool. People are 
beginning to experiment uses in a range of industries and corners of society. We have Siri on our Smartphones, robotics 
taking over as the manufacturing work horse, and computer diagnostic tools becoming smart enough to help us fix the 
very computer that they live on. Programs are more and more capable of reading in available information, digesting it 
and coming to a helpful conclusion. There are network diagnostic tools to figure out why your computer cannot connect
to the internet, and antivirus programs that analyze every inch of memory to snoop out malware, spyware, or even simple
corruption in the file structure. They do this multiple multitudes of times faster than their human predecessors and with
significantly increased efficiency. Computer programs don't become bored or fatigued- the only problem they run into is that 
they are as powerful and effective as the programmer that designed their algorithm. 

    \tab Diagnostic tools are at the point where problems such as network connectivity and search and find algorithms are 
child's play. They are faster and meaner than ever before and history, and AI researches are starting to experiment 
with ways of applying these tools to human diagnostics. A program can read in massive amounts of information and 
make sense of it, which is essentially what human doctors are paid and trained for years to do. There are 
recognizable patterns in the data that result in diagnoses. The patterns are distinguishable through medial histories 
and current symptoms, both are sources of information that computers can read from quickly and efficiently. The advantage 
of computer diagnostics are more readily seen when the source of information is so massive and complex that it would take 
a human specialist a long and costly time to find the pattern and resulting conclusion (in context of the medical field, 
a medication or treatment plan). An example would be in the form of  DNA sequencing and finding deeply hidden and 
complex diseases or cancers. 

    \tab Using computers to aid in the diagnostic and treatment in the human medical field will increase productivity and 
effectiveness overall. Of course programs and algorithms cannot be trusted completely for some time- the wrong diagnosis can
 be come to whether it's a human or a computer. The speed and sheer scaling power of using computers is an advantage that
 the medical field needs to take advantage of. Power artificial intelligence can aid in medical devices such an ultrasounds,
 aid in diagnoses of complex genetic diseases that would otherwise take a highly trained team of specialists, and even provide
 foresight and prevention of human mistakes in diagnosis.  

\section{Computer Aided Medical Tools}

\tab Humanity is by no means close to losing an entire profession. Doctors are still absolutely necessary for the health of 
a population, especially since production of AI and their respective robotics that pass the Turing Test, and can thusly replicate
the feel of a human person, isn't within immediate site. Person to person contact is vastly preferred when it comes to health
care rather than talking to a screen, even though I believe that a program could simulate that relationship well enough to be
satisfactory. An advantage is algorithmically generated and thus highly personalized relationships with a medial device.

\tab Human doctors are here to stay, so I digress. Doctors rely on their instruments and tools to accurately provide information
so that they can make an accurate and well informed diagnosis. There is a moving trend towards high resolution, 3D imaging to
provide information in a non-invasive, accurate way ~\cite{Zhang:2002:PPU:514191.514232}. This method allows doctors to 
literally see what is going on within their patients. 3D imaging using ultrasound has no unwanted side-effects similar to 
radiology imaging and is incredibly accurate, especially in shallow areas ~\cite{Zhang:2002:PPU:514191.514232}. 

\subsection{Improved Imaging Analysis}

\tab Researchers are working on improving this technology using clusters of computers connected through a low latency network
to increase computing power and speed up algorithm time. Images, and virtually all graphic algorithms, are better calculated
in parallel because of the nature of the problem. By increasing calculation power, doctors can perform more detailed ultrasounds
of their patients and potentially see results in real time ~\cite{Zhang:2002:PPU:514191.514232}. 

\section{Genetic Problem Solving}

\tab 

\subsection{Quicker, Efficient Analysis}

\tab 

\subsection{Solving Problems}

\tab 

\section{Doctor Oversight}

\tab 

\subsection{Linguistic Expression}

\tab 

\subsection{Diagnostic Correctness}

\tab 


\end{doublespace}

\nocite{*}

\bibliographystyle{plain}
\bibliography{master}

\end{document}