\documentclass[11pt]{article}

\usepackage{setspace}
\usepackage[letterpaper,margin=1in]{geometry}
\usepackage[parfill]{parskip}
\newcommand{\tab}{\hspace*{2em}}

\title{Computational Medical Diagnosis}
\author{John Foley}
\date{10 April 2014}

\begin{document}
\maketitle

\thispagestyle{empty}

\begin{abstract}
  Computers are increasingly being used in the medical industry. They are used to provide structure and efficiency to record 
  management and patient organization to both hospitals and medical labs, but the intelligence and efficiency of computers
  is also advanced enough to be able to analyze massive quantities of information and produce intelligent, reasonable
  conclusions. In the context of human medical records and genetic code, conclusions can be diagnoses and care planning. Analysis of this information is
  currently conducted by one or more highly trained doctors or medical experts. The cost of these trained professionals
  is relatively high compared to the time it takes to perform the analysis per patient. Using computers, with advanced
  algorithm and high level cognitive science research, we are able to replicate the team of specialists and come to the same
  results in significantly less time and hard work, freeing those professionals to be able to perform harder problems.
\end{abstract}

\pagebreak

\begin{doublespace}
\section{Introduction}

    \tab Computational intelligence and capability is quickly becoming noticeable in today's society as a helpful tool. People are 
beginning to experiment with them in a range of industries and corners of society. We have Siri on our Smartphones, robotics 
taking over as the manufacturing work horse, and computer diagnostic tools becoming smart enough to help us fix the 
very computer that they live on. Programs are more and more capable of reading in available information, digesting it 
and coming to a helpful conclusion. There are network diagnostic tools to figure out why your computer cannot connect
to the internet, and antivirus programs that analyze every inch of memory to snoop out malware, spyware, or even simple
corruption in the file structure. They do this multiple multitudes of times faster than their human predecessors and with
significantly increased efficiency. Computer programs don't become bored or fatigued- the only problem they run into is that 
they are as powerful and effective as the programmer that designed their algorithm. 

    \tab Diagnostic tools are at the point where problems such as network connectivity and search and find algorithms are 
child's play. They are faster and more powerful than ever before and history, and AI researchers are starting to experiment 
with ways of applying these tools to human diagnostics. A program can read in massive amounts of information and 
make sense of it, which is essentially what human doctors are paid and trained for years to do. There are 
recognizable patterns in the data that result in diagnoses. The patterns are distinguishable through medial histories 
and current symptoms, both are sources of information that computers can read from quickly and efficiently. The advantage 
of computer diagnostics are more readily seen when the source of information is so massive and complex that it would take 
a human specialist a long and costly time to find the pattern and resulting conclusion (in context of the medical field, 
a medication or treatment plan). An example would be in the form of  DNA sequencing and finding deeply hidden and 
complex diseases or cancers. 

    \tab Using computers to aid in the diagnostic and treatment in the human medical field will increase productivity and 
effectiveness overall. Of course programs and algorithms cannot be trusted completely for some time- the wrong diagnosis can
 be come to whether it's a human or a computer. The speed and sheer scaling power of using computers is an advantage that
 the medical field needs to take advantage of. Power artificial intelligence can aid in medical devices such an ultrasounds,
 aid in diagnoses of complex genetic diseases that would otherwise take a highly trained team of specialists, and even provide
 foresight and prevention of human mistakes in diagnosis.  

\section{Computer Aided Medical Tools}

\tab Humanity is by no means close to losing an entire profession. Doctors are still absolutely necessary for the health of 
a population, especially since production of AI and their respective robotics that pass the Turing Test, and can thusly replicate
the feel of a human person, isn't within immediate site. Person to person contact is vastly preferred when it comes to health
care rather than talking to a screen, even though I believe that a program could simulate that relationship well enough to be
satisfactory. An advantage is algorithmically generated and thus highly personalized relationships with a medial device.

\tab Human doctors are here to stay, so I digress. Doctors rely on their instruments and tools to accurately provide information
so that they can make an accurate and well informed diagnosis. There is a moving trend towards high resolution, 3D imaging to
provide information in a non-invasive, accurate way ~\cite{Zhang:2002:PPU:514191.514232}. This method allows doctors to 
literally see what is going on within their patients. 3D imaging using ultrasound has no unwanted side-effects similar to 
radiology imaging and is incredibly accurate, especially in shallow areas ~\cite{Zhang:2002:PPU:514191.514232}. 

\subsection{Improved Imaging Analysis}

\tab Researchers are working on improving this technology using clusters of computers connected through a low latency network
to increase computing power and speed up algorithm time. Images, and virtually all graphic algorithms, are better calculated
in parallel because of the nature of the problem. By increasing calculation power, doctors can perform more detailed ultrasounds
of their patients and potentially see results in real time ~\cite{Zhang:2002:PPU:514191.514232}. 

\section{Genetic Problem Solving}

\tab One of the areas of medical diagnosis that computers can help the most is in cancer and genetic disorder treatment and
diagnosis. These illnesses are generally extraordinarily complex and hard to calculate treatments for, assuming that the 
human specialist needed to diagnosis and treat the problem was able to accurately do so. I should say hard for current
specialists, and very expensive. Computers have always provided an excellent means to create an expert system, but unfortunately
cancer treatment is so large and complex that the system has to be flexible and consider or weigh a staggering amount of 
variables ~\cite{Wang:2012:RCM:2122263.2122454}.

\tab Tumor classification based on Gene Expression Profiles (GEPs), which is of great benefit to the accurate diagnosis 
and personalized treatment for different types of tumor, has drawn a great attention in 
recent years. ~\cite{Wang:2012:RCM:2122263.2122454}. Because of the sheet amount of genetic material in the 
human genome, spotting health problems before they can even form is near impossible unless you know what you're looking
for. One possible avenue in biology is to create correlation filters to find patterns in genetic material, no matter what
different gene type they happen to be expressed in. Minimum Average Correlation Energy (MACE) and Optimal Tradeoff Synthetic
Discriminant Function (OTSDF), are introduced to determine whether a test sample matches the templates synthesized for 
each subclass ~\cite{Wang:2012:RCM:2122263.2122454}. The template represents a known genetic pattern for a tumor type, 
and the test sample represents up to an entire
gene (which contains an immense amount of genetic information). Each correlation filter attempts to take advantage of a
computer's processing power to determine the correlation of a gene and a known tumor pattern.

\tab Analyzing genetic information is further complicated by almost never having identically matching patterns. The lengths of 
genetic material that represents the testing samples are produced in chunks, and so test samples may overlap and mismatch. MACE
and OTSDF have shown progress in being able to detect similarity of the overall pattern while ignoring common mismatches
~\cite{Wang:2012:RCM:2122263.2122454}. 

\subsection{Quicker, Efficient Analysis}

\tab Computer intelligence armed with these power filtering algorithms have produced incredible results. They are able to find 
overall similar patterns in half the time it would take a team of geneticists to sift through results. Not only is the batch 
operation time better, but applications are optimizing the process by finding similarities in the smallest amount of genes 
possible ~\cite{Wang:2007:ACC:1229968.1229975}. The significance of this finding results in the ability to extract simple
diagnostic rules to diagnose a problem without the need for classifiers. The largest, and most clear benefit is that now
problems can be found with relatively few gene expressions rather than thousands to have to filter through. 

\tab Many types and subtypes of cancer can be linked directly to DNA mutations or corruption, so being able to find the
problematic gene patterns early can reduce the risk of developing a large amount of lethal cancer cells. While prevention
is preferred, treatment is obviously of high importance also. Treatment is very difficult because of all the variables that 
come into the play: diet, bodily reaction, medications, and underlying genetic conditions. A cancer treatment center has 
recently been developing a computer system that optimizes a simulation to accurately piece together effective treatment 
solutions ~\cite{Baesler:2001:HIM:564124.564329}. 

\subsection{Solving Problems}

\tab There are three primary reasons to use computers to aid in diagnosis and treatments in the medical industry. They reduce
errors, provide economy and efficiency, and quantitative quality of probability ~\cite{Yarnall:1966:CAM:800256.810706}. One of 
the obvious problems with integrating computers into the diagnosis and treatment planning cycle is if physican to computer
interaction is scarce, and the general limitation of hardware for such a daunting data processing task. The use of computers
for record storage and retreival is immediately feasable, but the ability to process complicated biological and mathematical
models concerned with probabilty and symptom to disease matching, which exists inherently in the problem, may not be within
the common hospital's or medical lab's reach ~\cite{Yarnall:1966:CAM:800256.810706}. 

\tab Another problem to solve is the interaction between humans and machines. Person to terminal interaction, as discussed 
before, is a challenge to overcome because of a person's ability to spot a fake human ~\cite{Barnett:1987:HDM:41526.41531}. 
These problems are easy to overcome, but will have to come with the passage of time and the further integration of interactive
technology with every day life.  

\section{Diagnostic Decision Making}

\tab Computer expert systems are being used in the medical field more and more. They are being used by doctors to help make diagnostic decisions, but as increasingly being seen, giving the medical diagnosis as well. Decision making artificial intelligence is advancing to the point where learning based on test cases is relatively easy. A challenge with using computers is that they are very good at coming up with a final decision, but lack the ability to give argumentative support as to why or how they came to that conclusion. At least in a way that the user can understand, and that is a meaningful problem.

\tab Financial systems and problems are similar to medical diagnostics. Both involve accepting huge amounts of variables and attempting to come it a conclusion that explains what caused the variables to come to the result that they did. Researchers in Oregon studied fiancial systems' decision making, and how it is similar to human decision making. They figured out the discrete algorithms that the student's came to by asking them to talk out loud during their decision process. The researchers attempted to mimic this effect with the computer systems, having them 'talk' out loud as they computed. This obviously didn't work as well because of linguistic rigidity, but it shed some light to the decision making process.

\tab This technique struck me as an effective way to test computer medical diagnostic systems as well. As doctors come to trust output from computers, whether they be a fledged out system or the doctors' tools, they will need to know how the computers came their conclusions. Some sort of traceable output to map the path to the answer. Unfortunately, the same challenge that the researchers in Oregon will present itself here as well. Computers cannot voice their various steps as well as a financial student, but as interactive systems advance this become a minor problem.

\section{Trusting Solutions}



\section{Conclusion}

\tab Human illness is a very challenging problem to figure out, despite the patterns associated with each disease. The
immensity of the information associated with the process, not to mention the enormity of the database held in DNA, has 
lead to a coming paradigm shift of using computers as diagnostic machines for humans. Analyzing these patterns and using
probably models and DNA sequencing to calculate the most effective treatment plans is increasingly being done by 
artificially intelligent machines. Computers are less likely to commit preventable errors, and have vastly increased 
economy and efficiency in the diagnostic process. IBM's Watson was just recently released onto genome data to detect 
cancer in patients, soon this won't be news but everyday events. 

\pagebreak


\end{doublespace}

\nocite{*}

\bibliographystyle{plain}
\bibliography{master}

\end{document}